\chapter{Introduction}
\label{chapter:introduction}

In this chapter some background will be given for the problem statement, as well as the motivation and objectives behind this work.

\section{Background}
    Skin cancer is the out-of-control growth of abnormal cells in the outermost skin layer, caused by mutations triggered within the genetic code \cite{Foundation2019}. These mutations lead the skin cells to multiply rapidly and form malignant tumors. It is currently the most common type of cancer \cite{Rogers2015} with melanoma being the most deadly form that accounts for about 75\% of skin cancer deaths, even though it only represents 5\% of all skin cancer cases \cite{Esteva2017}. Other forms of skin cancer such as the \ac{BCC} or the \ac{SCC} are not as deadly but are much more common. \ac{BCC}s are the most common type of skin cancer and have the potential to disfigure the skin and become dangerous \cite{Rogers2015}. Similarly, \ac{SCC}s are the second most common form of this decease and, if left untreated, can destroy nearby healthy tissue, spread to other parts of the body, and eventually lead to death \cite{Foundation2019}. \par
    
    A recent study observed that in the United States of America alone, there are 5.4 Million new cases of skin cancer every year \cite{Rogers2015}. However, a bigger problem is that the incidence rates of skin cancer keep rising with currently 1 in 5 persons developing skin cancer until the age of 70 \cite{Foundation2019}. In Europe, melanoma incidence rates also manifest heavily, as each year 100000 people are diagnosed with melanoma, and approximately 22000 people die annually from this form of skin cancer \cite{Bray2018}. \par 
    
    Even though skin cancer can be deadly in late stages, if detected early there is a high chance of survival. For example, there is a 23\% chance of surviving a melanoma case if detected in the late stages, but if detected early, the five-year survival rate (\textit{i.e.}, percentage of people alive after five years) is approximately 98\% \cite{Foundation2019}. Therefore, the early detection of skin cancer is a top priority in order to increase the overall survival rate of the population. \par
    
    Skin cancer can be detected by dermatology professionals by performing a visual examination of skin lesions. However, some authors believe that the dermatologist's experience directly impacts his diagnostic accuracy \cite{Haenssle2018}, which implies that different physicians might make a different diagnosis for the same lesion. Some works have corroborated this in practice. For example, Argenziano \textit{et al.} in their study noted that dermatologists have a 65\% to 80\% interval of accuracy in melanoma diagnosis \cite{Argenziano2001}. However, this rate can be overall increased with the supplement of dermatoscopic images \cite{Kittler2002}. These type of images are taken with a special high-resolution and magnifying camera in a controlled light environment, where reflections on the skin are minimized which causes deeper skin layers to be visible. \par

    Automated diagnosis of skin lesions through the dermatoscopic and non-dermatoscopic images has been achieving significant progress over the years, as demonstrated by some state-of-the-art surveys \cite{Okur2018}\cite{Pathan2018}\cite{Mishra2016}. Typically, computer vision algorithms are used to analyze images and extract structural information \cite{Henning2007}, such as the ABCDE (Asymmetry, Border, Color, Dermoscopic structure, and Evolving) rule \cite{Nachbar1994} or the CASH (color architecture, symmetry, and homogeneity) algorithm. These algorithms are then integrated within systems to provide quantification of features for physician assessment and for provisional diagnosis. \par
    
    However, recent studies show that more complex structures called Convolutional Neural Networks (\ac{CNN}s) are being applied to classify skin lesions with remarkable performance in comparison to human experts \cite{Esteva2017}\cite{Haenssle2018}\cite{Tschandl2019}. Moreover, the interest on this topic is well demonstrated by the large number of papers related to deep learning published in the context of the \ac{ISIC} challenges \cite{humanvsisic2018}. This paradigm shift became apparent in 2016, for the International Symposium on Biomedical Imaging (ISBI) benchmark challenge towards melanoma detection. From 25 teams, all of them employed \ac{CNN}s, instead of hand-crafted computer vision algorithms or other machine learning methods \cite{Brinker2018}. 
    
    \ac{CNN}s are a type of deep learning architecture. In turn, deep learning refers to computational models composed of multiple processing layers capable of learning representations of data with multiple levels of abstraction \cite{Goodfellow-et-al-2016}. Deep learning topologies have a considerable advantage over traditional Artificial Neural Networks (\acs{ANN}s), namely, they do not require the selection of an appropriate set of features from which the algorithm learns. This task involves extracting knowledge or information which is implicit in the raw data, requiring careful engineering and knowledge of the problem domain. Some developments allowed \ac{CNN}s to emerge as an feasible approach to computer vision classification problems, such as the use of the \ac{GPU} to speedup computations and the development of high-level software modules to train deep neural networks. \par
    
\section{Motivation}
    Current implementations of deep learning models for skin lesion diagnosis still present major challenges to overcome. The biggest one is the requirement for large amounts of training data in order to achieve superior performance to other methods. However, publicly available datasets for medical imaging related problems are typically small compared to the needs of training a deep \ac{CNN} from scratch \cite{Ching2018}. Even though training a model from scratch is a feasible approach for big datasets (\textit{e.g.}, ImageNet \cite{Deng2010}), alternative approaches like transfer learning present an workaround for small datasets. Transfer learning is a technique in which one can leverage the knowledge obtained from a deep model previously trained on a large labelled dataset towards a new classification task \cite{Yosinski2014}. Furthermore, transfer learning does not require as much machine learning expertise as designing a model from scratch, tuning its hyperparameters, and making thorough discussions about the results. For this reason, in recent years, transfer learning based approaches have become prevalent for automated skin lesion diagnosis \cite{Esteva2017}\cite{Haenssle2018}\cite{gessert2018}\cite{isic2019first}. 
    
    However, some questions remain open in the context of transfer learning based models for skin lesion classification, namely: "\textit{Which strategy should one chose to train these transfer learning based models?}", "\textit{What pre-trained models and architectures bring benefits to the classification of skin lesions?}", "\textit{How does model depth affects the performance of transfer learning based deep learning models?}", and "\textit{How can hyperparameter optimization help transfer learning based approaches achieve higher generalization performance?}". \par 
    
    In addition to transfer learning, a commonly used method to ease the requirement for large amounts of data is data augmentation \cite{Perez2018}. This term involves a large range of methodologies which aim to create synthetic samples that allows a model to attain better generalization performance. Furthermore, skin lesion datasets are often highly unbalanced which can cause the model to optimize its results for overrepresented classes. Several approaches attempt to solve this issue by class balancing samples through data augmentation \cite{He2009}. However, some questions remain open about its usage, namely: "\textit{Can class balancing through data augmentation help increase generalization performance of underrepresented classes?}", "\textit{What is the influence of dataset size on generalization performance?}", "\textit{What is the influence of data augmentation in deep learning for small datasets?}", "\textit{Which data augmentation image processing algorithms should one apply for the problem of skin lesion classification?}". \par
    
    Moreover, the top approaches towards skin lesion classification systems of benchmark challenges such as the \ac{ISIC} 2019 \cite{isic2019}, usually employ model ensembling methods in order to attain better generalization performance \cite{isic2019first} \cite{isic2019second} \cite{isic2019third}. However, one could ask the following open questions: "\textit{How can one employ these type of techniques to get meaningful improvements on the generalization performance?}", "\textit{Do the obtained improvements justify the disadvantages of using these types of methods?}", and "\textit{Are these approaches practical to be deployed into the clinical workflow?}". Ultimately, the goal of all of these questions is to further improve generalization performance of deep learning models. \par
    
    However, in a production environment, samples of skin lesions may be taken under far different conditions from their training dataset (\textit{e.g.}, lighting conditions) or might not be part of the original training categories. Work in this field reports that deep learning models are highly sensitive to a different sample distribution from their original training distribution \cite{humanvsisic2018}\cite{Han2018}. This is a known problem of deep learning models, which out-of-distribution detection methods attempt to solve. Ideally, these types of methods could flag out-of-distribution samples for further inspection of a dermatologist, which would ultimately augment his capabilities, rather than replace them. However, some questions remain open towards this methods in the context of skin lesion classification: "\textit{What methods can one use to detect out of training distribution samples?}" and "\textit{What is the effectiveness of these methods for skin lesion diagnosis?}".

\section{Objectives}
    A strong assumption can be made that the recent advancements of deep learning based methods have the potential to change the landscape of skin lesion diagnosis because it would minimize the time it takes to detect skin cancer cases. However, one must first closely study the impact of the presented problems and the different strategies to deal with them, as it will likely lead to useful insights and contributions to the current state-of-the-art methods. Therefore, the following key objectives are highlighted for this work:
    \begin{itemize}
        \item Study the impact of transfer learning methodologies, pre-trained model architectures and parameters in the context of skin lesion classification;
        \item Study the impact of data augmentation, class balancing, and ensembling techniques as methods to improve performance for skin lesion classification;
        \item Experiment with different strategies to identify out of training distribution samples from the test set;
        \item Create a comprehensive approach for the \ac{ISIC} 2019 benchmark challenge that provides some insight into ways of optimizing generalization performance.
    \end{itemize}

\section{Outline}
    Considering the presented objectives, the remainder of this dissertation is organized as follows:
    \begin{itemize}
        \item \Cref{chapter:mam} introduces the reader to deep learning concepts and techniques relevant to the current state-of-the-art methods which automate skin lesion diagnosis using deep learning;
        \item \Cref{chapter:sota} provides an overview of recent progress in the automated diagnosis of skin lesions;
        \item \Cref{chapter:environment} describes the experimental scope of this work and the setup used for the experiments; 
        \item \Cref{chapter:experiments} presents a range of experiments with different pre-trained models, transfer learning methods and hyperparameters; 
        \item \Cref{chapter:experiments2} studies the impact of dataset size, data augmentation methods, ensembling techniques and methodologies to detect out of training distribution samples;
        \item \Cref{chapter:conclusion} offers final remarks, key takeaways, limitations of this research, and directions for future work.
    \end{itemize}